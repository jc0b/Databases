%& -job-name=Databases
\documentclass{article}
\usepackage[utf8]{inputenc}
\usepackage{listings}
\usepackage{float}


\title{Databases Summary 2018}
\author{Jacob Burley}
\date{Block 5 2018}

\begin{document}
%TODO: organisation
%       questionable/unquestionable functional dependency
\long\def\/*#1*/{}
\restylefloat{table}
\maketitle

\section*{Lecture 1}
\subsection*{Transactions}
A transaction is a collection of operations that performs a single \textbf{logical} function within a database application.
\\The \textbf{Transaction Management Component} ensures that the database remains in a consistent (aka correct) state, even if a \textbf{system failure} (e.g. power loss, OS crash etc) or a \textbf{transaction failure} occurs.
\\We also have a \textbf{Concurrency Control Manager} that controls the interactions between any concurrent transactions, helping to make sure that the database remains in a consistent state.

\subsection*{ACID properties}
These properties of a Database Management System are critical in order to maintain the consistency of the database, as mentioned above.
\begin{itemize}
    \item \textbf{Atomicity:} The transaction either is completed (it commits), or it doesn't impact the database at all (it aborts)
    \item \textbf{Consistency:} The transaction always leaves the database in a consistent state. This means that all defined \textbf{integrity constraints} hold.
    \item \textbf{Isolation:} Multiple users can modify the same database concurrently, but their transactions will not be visible to other users while incomplete. Essentially, the final system state is the same as what would be obtained if all the transactions would be executed in sequence.
    \item \textbf{Durability:} Once a transaction has committed successfully, the modified data persists, even if there is a system failure such as a disk crash.
\end{itemize}
\newpage    
\section*{Lecture 2}
\subsection*{Relational Model}
Also known as "rectangular tables"
\begin{itemize}
    \item The \textbf{Schema} is the structure of the database, which is the same as the \textbf{relations + constraints}.
    \item A \textbf{database} is a set of named \textbf{relations} (\textbf{tables})
    \item Each relation in a database has a set of named \textbf{attributes} (\textbf{columns})
    \item Each \textbf{tuple} (\textbf{row}) has a value for each attribute
    \begin{itemize}
        \item A tuple may contain NULL values, indicating that a value is unknown or undefined.
        \item NULL values can be disallowed by the datatype of the attribute, which would mean that attempting to insert a tuple containing a null value would result in failure. For example, if we wanted a \verb|VARCHAR| field to not allow null, we would declare it as \verb|VARCHAR(20) NOT NULL| in SQL.
        \item NULL values being disallowed can lead to users inventing fake values to fill missing columns. This is bad as it can, for example, affect data analysis done on this data in the future.
        \item NULL is \textbf{not} the same as the number 0, and is also \textbf{not} the same as an empty string. It is its own datatype.
        \item There is no clear definition of how to use null. SQL uses \textbf{three-valued logic} (true, false, unknown) for the evaluation of comparisons involving null values.
    \end{itemize}
    \item Each attribute has a \textbf{datatype} (\textbf{domain})
    \item A \textbf{key} is an attribute whose value is unique in each tuple.
    \begin{itemize}
        \item A key can also be a set of attributes whose combined values are unique
        \item We can indicate a key in a schema by \underline{underlining} the attribute(s) that make(s) up the key
    \end{itemize}
\end{itemize}
\subsubsection*{Creating a table using a relation schema}
\begin{table}[H]
\begin{center}
 \begin{tabular}{| c | c | c | c |} 
 \hline
 \underline{\texttt{CAT}} & \underline{\texttt{ENO}} & \texttt{TOPIC} & \texttt{MAXPT} \\ [0.5ex] 
 \hline
 H & 1 & Rel.Alg. & 10 \\
 H & 2 & SQL & 10 \\
 H & 1 & SQL & 14 \\
 \hline
\end{tabular}
\end{center}
\caption{\texttt{EXERCISES}}
\end{table}
The relation schema for the above table is expressed as:
\begin{center}
    Exercises(\underline{CAT}, \underline{ENO}, TOPIC, MAXPT)
\end{center}
We can then express this as a SQL table creation query as below:
\begin{lstlisting}
CREATE TABLE EXERCISES (CAT    CHAR(1),
                        ENO    NUMERIC(2),
                        TOPIC  VARCHAR(40),
                        MAXPT  NUMERIC(2));
\end{lstlisting}

\subsection*{Constraints on databases}
A database is intended to represent a \textbf{subset} of the real world. It is \textbf{not} meant to represent every conceivable state of the real world, as there are infinitely many states the world could be in. This means that we intend to reduce the number of states that the database can be in, removing meaningless and or illegal states.
\\We restrict the set of possible database sets using \textbf{Integrity Contraints (IC)}. With ICs, we hope to only allow images of \underline{possible} real world scenarios. We specify our ICs as part of the database schema.
\\The DBMS will \textbf{refuse any update} which would lead to a database state that violates any of the constraints.































\/*
\section*{Relational Models \& Algebra}
Schema in a database: structure of the DB = relations + constraints
\\For example, for a Customer(id number, name, street, city), we can say that the primary key constraint is on the id number. It can be used as the primary key, as the id should be unique, whereas multiple customers can share the same name, street or address
\\We can use constraints in different ways. We can constrain the datatypes for a given field, for example an id number should not allow letters.  
\subsection*{Relation Schema}
A relation schema $s$ (a.k.a the schema of a single relation) defines the following:
\begin{itemize}
    \item a (finite) sequence $A_1,...,A_n$ of \textbf{attribute names}
    \begin{itemize}
        \item these names must be distinct
        \item i.e. $A_i \neq A_j$ for $i \neq j$
    \end{itemize}
    \item For each attribute $A_i$, a data type (or \textbf{domain}) $D_i$
    \begin{itemize}
        \item for each A, we let $domain(A_i):= value(D_i)$
    \end{itemize}
    \item We can write this schema as $$s = (A_1 : D_1,...,A_n : D_n)$$
\end{itemize}

\section*{Shit we can do with tables}
\subsection*{Join}
\subsubsection*{Inner}
Returns records that have matching values in both tables
\begin{table}
\begin{center}
 \begin{tabular}{|c c c c|} 
 \hline
 Col1 & Col2 & Col3 & Col4 \\ [0.5ex] 
 \hline\hline
 1 & 6 & 87837 & 787 \\ 
 \hline
 2 & 7 & 78 & 5415 \\
 \hline
 3 & 545 & 778 & 7507 \\
 \hline
 4 & 545 & 18744 & 7560 \\
 \hline
 5 & 88 & 788 & 6344 \\ [1ex] 
 \hline
 
\end{tabular}
\end{center}
\caption{table1}
\end{table}
\begin{table}
\begin{center}
\begin{tabular}{|c c c c|} 
 \hline
 Col1 & Col2 & Col5 & Col6 \\ [0.5ex] 
 \hline\hline
 14 & 62 & 837 & 77 \\ 
 \hline
 22 & 723 & 7854 & 54 \\
 \hline
 312 & 55 & 7748 & 717 \\
 \hline
 454 & 5 & 1384 & 60 \\
 \hline
 534 & 658 & 7348 & 623144 \\ [1ex] 
 \hline
 
\end{tabular}
\end{center}
\caption{table2}
\end{table}
\begin{lstlisting}[language=SQL]
SELECT table1.Col1, table2.Col5, table1.Col3 
FROM Orders 
INNER JOIN table1 ON table1.Col1=table1.Col1;
\end{lstlisting}
\subsubsection*{Left Outer}
\subsubsection*{Right Outer}
\subsubsection*{Full Outer}
*/
\end{document}
